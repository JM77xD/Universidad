%%%%%%%%%%%%%%%%%%%%%%%%%%%%%%%%%%%%%%%%%
% Short Sectioned Assignment LaTeX Template Version 1.0 (5/5/12)
% This template has been downloaded from: http://www.LaTeXTemplates.com
% Original author:  Frits Wenneker (http://www.howtotex.com)
% License: CC BY-NC-SA 3.0 (http://creativecommons.org/licenses/by-nc-sa/3.0/)
%%%%%%%%%%%%%%%%%%%%%%%%%%%%%%%%%%%%%%%%%

%----------------------------------------------------------------------------------------
%	PACKAGES AND OTHER DOCUMENT CONFIGURATIONS
%----------------------------------------------------------------------------------------

\documentclass[paper=a4, fontsize=11pt]{scrartcl} % A4 paper and 11pt font size

% ---- Entrada y salida de texto -----

\usepackage[T1]{fontenc} % Use 8-bit encoding that has 256 glyphs
\usepackage[utf8]{inputenc}
%\usepackage{fourier} % Use the Adobe Utopia font for the document - comment this line to return to the LaTeX default

% ---- Idioma --------

\usepackage[spanish, es-tabla]{babel} % Selecciona el español para palabras introducidas automáticamente, p.ej. "septiembre" en la fecha y especifica que se use la palabra Tabla en vez de Cuadro

% ---- Otros paquetes ----

\usepackage{url} % ,href} %para incluir URLs e hipervínculos dentro del texto (aunque hay que instalar href)
\usepackage{amsmath,amsfonts,amsthm} % Math packages
%\usepackage{graphics,graphicx, floatrow} %para incluir imágenes y notas en las imágenes
\usepackage{graphics,graphicx, float} %para incluir imágenes y colocarlas

% Para hacer tablas comlejas
%\usepackage{multirow}
%\usepackage{threeparttable}

%\usepackage{sectsty} % Allows customizing section commands
%\allsectionsfont{\centering \normalfont\scshape} % Make all sections centered, the default font and small caps

\usepackage{fancyhdr} % Custom headers and footers
\pagestyle{fancyplain} % Makes all pages in the document conform to the custom headers and footers
\fancyhead{} % No page header - if you want one, create it in the same way as the footers below
\fancyfoot[L]{} % Empty left footer
\fancyfoot[C]{} % Empty center footer
\fancyfoot[R]{} % Page numbering for right footer
\renewcommand{\headrulewidth}{0pt} % Remove header underlines
\renewcommand{\footrulewidth}{0pt} % Remove footer underlines
\setlength{\headheight}{13.6pt} % Customize the height of the header

\numberwithin{equation}{section} % Number equations within sections (i.e. 1.1, 1.2, 2.1, 2.2 instead of 1, 2, 3, 4)
\numberwithin{figure}{section} % Number figures within sections (i.e. 1.1, 1.2, 2.1, 2.2 instead of 1, 2, 3, 4)
\numberwithin{table}{section} % Number tables within sections (i.e. 1.1, 1.2, 2.1, 2.2 instead of 1, 2, 3, 4)

\setlength\parindent{0pt} % Removes all indentation from paragraphs - comment this line for an assignment with lots of text

\newcommand{\horrule}[1]{\rule{\linewidth}{#1}} % Create horizontal rule command with 1 argument of height


\usepackage{hyperref}
\usepackage{listings}
\usepackage{xcolor}
\usepackage{caption}
\usepackage{subcaption}
\usepackage{lmodern}
\usepackage{graphicx}
\usepackage{biblatex}
\usepackage{array}
\usepackage{wrapfig}

\definecolor{codegreen}{rgb}{0,0.6,0}
\definecolor{codegray}{rgb}{0.5,0.5,0.5}
\definecolor{codepurple}{rgb}{0.58,0,0.82}
\definecolor{backcolour}{rgb}{0.95,0.95,0.92}

\newcolumntype{C}[1]{>{\centering\arraybackslash}p{#1}}
\newcolumntype{L}[1]{>{\raggedright\arraybackslash}p{#1}}
\newcolumntype{R}[1]{>{\raggedleft\arraybackslash}p{#1}}

\lstdefinestyle{mystyle}{
    backgroundcolor=\color{backcolour},   
    commentstyle=\color{codegreen},
    keywordstyle=\color{magenta},
    numberstyle=\tiny\color{codegray},
    stringstyle=\color{codepurple},
    basicstyle=\ttfamily\footnotesize,
    breakatwhitespace=false,         
    breaklines=true,                 
    captionpos=b,                    
    keepspaces=true,                 
    numbers=left,                    
    numbersep=5pt,                  
    showspaces=false,                
    showstringspaces=false,
    showtabs=false,                  
    tabsize=2
}


\title{	
\normalfont \normalsize 
\textsc{\textbf{Técnicas de los Sistemas Inteligentes grupo 2 (2021-2022)} \\ Grado en Ingeniería Informática \\ Universidad de Granada} \\ [25pt] % Your university, school and/or department name(s)
\horrule{0.5pt} \\[0.4cm] % Thin top horizontal rule
\huge Práctica 2 \\
MiniZinc \\ % The assignment title
\horrule{2pt} \\[0.5cm] % Thick bottom horizontal rule
}

\author{José María Ramírez González\\\href{mailto:jmramirez@correo.ugr.es}{jmramirez@correo.ugr.es}} % Nombre y apellidos

\date{\normalsize\today} % Incluye la fecha actual

\begin{document}

\maketitle % Muestra el Título

\newpage %inserta un salto de página

\tableofcontents % para generar el índice de contenidos

\newpage

\section{Introducción}

En el desarrollo de esta práctica vamos a trabajar con \href{https://www.minizinc.org/}{MiniZinc}, un software que resuelve problemas de satisfacción de restricciones.

Vamos a explorar cinco problemas, en los que abordaremos varias preguntas relativas a cada uno de ellos.

Estos problemas son:

\begin{itemize}
    \item Problema de las monedas.
    \item Problema de los horarios.
    \item Un problema lógico.
    \item Problema de asignación de tareas.
    \item Problema de coloreado de grafos.
\end{itemize}

Entraremos más en detalle sobre cada problema en la sección del mismo.

\newpage

\section{Problema de las monedas}

Este problema consiste en encontrar un conjunto de monedas cuyo importe sea exactamente una cantidad dada.

Tenemos disponibles todas las monedas disponibles aquí en España, tanto céntimos, como euros.

Una vez modelado el problema, no vamos a encontrar la solución óptima (menor número de monedas), si no que vamos a encontrar todas las posibles soluciones para las cantidades de 0.17€, 1.43€, 2.35€ y 4.99€. También anotaremos la primera solución encontrada y el tiempo total en encontrar las soluciones del problema.


\begin{tabular}{|c|p{0.3\textwidth}|p{0.3\textwidth}|c|}
    \hline
    Importe & Primera solución encontrada y número de monedas en la misma & Número total de soluciones & Runtime (segundos) \\
    \hline 0.17 € & 17 monedas de 1 céntimo, total 17 monedas & 28 & 0.061\\
    \hline 1.43 € & 143 monedas de 1 céntimo, total 143 monedas & 17952 & 2.142\\
    \hline 2.35 € & 235 monedas de 1 céntimo, total 235 monedas & 150824 & 21.469 \\
    \hline 4.99 € & 499 monedas de 1 céntimo, total 499 monedas &  & \\
    \hline
\end{tabular}


Como vemos, no nos ofrece soluciones nada óptimas y el tiempo de ejecución crece de forma exagerada con la cantidad. Vamos a añadir restricciones para que la parte entera del importe se asigne únicamente a monedas de 1 y 2 euros.

Vamos a realizar una nueva tabla igual a la anterior a ver si hemos conseguido mejorar los resultados.

\begin{tabular}{|c|p{0.3\textwidth}|p{0.3\textwidth}|c|}
    \hline
    Importe & Primera solución encontrada y número de monedas en la misma & Número total de soluciones & Runtime (segundos) \\
    \hline 0.17 € & 17 monedas de 1 céntimo, total 17 monedas & 28 & 0.057\\
    \hline 1.43 € & 43 monedas de 1 céntimo y 1 moneda de 1 euro, total 44 monedas & 284 & 0.104\\
    \hline 2.35 € & 35 monedas de 1 céntimo y 1 moneda de 1 euro, total 36 monedas & 162 & 0.096 \\
    \hline 4.99 € & 99 monedas de 1 céntimo y 2 monedas de 2 euros, total 101 monedas & 4366 & 0.509 \\
    \hline
\end{tabular}


Como vemos, hemos mejorado considerablemente el tiempo de resolución y los resultados obtenidos, no obstante, seguimos sin tener la solución óptima. Para esto, tenemos que minimizar el número de monedas a obtener cambiando \texttt{solve satisfy} por \texttt{solve minimize cantidadMonedas}\footnote{En nuestro caso hemos llamado \textit{cantidadMonedas} a la variable que almacena el número total de monedas}

Vamos a generar una nueva tabla, esta vez con las soluciones óptimas, la cantidad de monedas de las mismas, y el tiempo de ejecución.

\begin{tabular}{|c|p{0.5\textwidth}|c|c|}
    \hline
    Importe & Solución óptima & Número de monedas & Runtime (s)\\
    \hline 0.17 € & 1 moneda de 10 cent, 1 moneda de 5 cent y 1 moneda de 2 cent & 3 & 0.054\\
    \hline 1.43 € & 1 moneda de 1 euro, 2 monedas de 20 cent, 1 moneda de 2 cent y 1 moneda de 1 cent & 5 & 0.053\\
    \hline 2.35 € & 1 moneda de 2 euros, 1 moneda de 20 cent, 1 moneda de 10 cent y 1 moneda de 5 cent & 4 & 0.053\\
    \hline 4.99 € & 2 monedas de 2 euros, 1 moneda de 50 cent, 2 monedas de 20 cent,1 moneda de 5 cent y 2 monedas de 2 cent & 8 & 0.057\\
    \hline
\end{tabular}

Ahora si, con esta última modificación obtenemos soluciones óptimas.

Se nos plantean ahora las siguientes preguntas:

\textbf{¿Qué ocurriría si usando la primera codificación, el importe buscado es mucho mayor?}

Pues bien, si el importe buscado fuera exageradamente más grande que 4.99€, tendríamos un problema, ya que el número de combinaciones posibles sería enorme, por lo que el tiempo de ejecución y la memoria que consumiría el programa se volvería impracticable. Es decir, podríamos ejecutar el programa, pero este no acabaría nunca.

\textbf{¿Se podría encontrar alguna solución usando la primera codificación con una cantidad del orden de millones de euros? ¿Cuál sería una estrategia prometedora para esto?}

Realmente, podríamos encontrar una solución usando la primera codificación, el problema reside en que no podríamos encontrarlas todas, ya que en términos de tiempo y memoria sería imposible. No obstante, si limitamos el número de soluciones que queremos encontrar a ``1'', si que encontraríamos la primera solución disponible\footnote{Por cómo está diseñada la primera codificación, el conjunto solución solo tendría millones de monedas de 1 céntimo}, pero esta no sería ni siquiera cercana a la óptima.

\newpage

\section{Problema de los horarios}

En este problema se nos pide encontrar una asignación de horarios que satisfaga unas condiciones dadas.

Con una primera implementación sencilla obtendríamos algunas posibles soluciones.


Se nos plantean ahora las siguientes preguntas:

\textbf{¿Existen soluciones simétricas?}

Si, existen soluciones simétricas con la codificación inicial, ya que ahora mismo, esta codificación no comprende que, por ejemplo, si tenemos una asignatura x y 2 horas seguidas, x\_1 y x\_2, es lo mismo tener x\_1 y x\_2 que tener x\_2 y x\_1. Aunque sean horas distintas, al ser de la misma asignatura, da igual y, por tanto, semánticamente son lo mismo.

\textbf{¿Cómo se podrían evitar las soluciones simétricas y cuál sería el número de soluciones no simétricas?}

Se podría evitar añadiendo nuevas restricciones que tuvieran en cuenta el número de horas que tiene una asignatura y no las horas como concepto propio, es decir, que contaran el número de horas de asignatura y no asignaran la franja a una hora concreta de la asignatura, si no a la asignatura propiamente. Usando esta codificación obtendríamos un menor número de posibles soluciones.
En el fichero MZN se puede observar comentado el código del que hemos prescindido y se han señalado las nuevas restricciones.

\newpage

\section{Problema lógico}

En este problema, en el que se nos presentan cinco personas, cada una de un lugar y con unos gustos concretos, se nos pide encontrar de quién es un animal concreto(cebra) y quién bebe una bebida concreta (agua).

Aparentemente, no hay una solución directa, pues tenemos un gran número de condiciones que tenemos que comprobar antes de suponer una cosa u otra en base a la información inicial.

Finalmente, tras implementar todas las restricciones dadas, podemos afirmar la respuesta a la pregunta que se nos plantea.

\textbf{¿Dónde está la cebra y quién bebe agua?}

Pues bien, obtenemos que la cebra está en la casa del gallego y que la persona que bebe agua es el andaluz.

\newpage

\section{Problema de asignación de tareas}

En este problema, tenemos disponibles tres trabajadores y una lista de 9 tareas a realizar, cada trabajador tarda un tiempo determinado en realizar cada tarea. Además hay tareas que tienen que realizarse antes de que puedan comenzar otras.

Tenemos que buscar una asignación de tareas que minimice el tiempo en realizar todas las tareas.

Se nos plantea la cuestión de \textbf{¿Cuál es la duración mínima de construcción de la casa?}

Pues bien, con el problema planteado como hemos dicho, tardaremos un total de 11 días, con una asignación de tareas, siendo 1 el trabajador uno y 3 el trabajador 3 como sigue:

\begin{center}
\begin{tabular}{|c|c|c|c|}
    \hline
    Tarea & Día inicio & Día final & Trabajador asignado\\
    \hline A & 0 & 4 & 1\\
     B & 4 & 7 & 1\\
     C & 7 & 8 & 2\\
     D & 7 & 9 & 1\\
     E & 9 & 11 & 2\\
     F & 9 & 10 & 3\\
     G & 9 & 10 & 1\\
     H & 4 & 7 & 2\\
     I & 10 & 11 & 1\\
    \hline
\end{tabular}
\end{center}

\newpage

\section{Problema de coloreado de grafos}

En este problema se nos pide colorear las aristas de un grafo con el menor número de colores posible tal que no haya dos aristas que compartan un nodo que tengan el mismo color.
A su vez, nos tenemos que asegurar que aquellas aristas que sean similares (es decir, que unan los mismos nodos) tengan el mismo color.

Vamos a rellenar la siguiente tabla con el número de nodos(N), el número de aristas(M), el tiempo total en segundos y el número mínimo de colores necesarios para colorearlo.

Hemos ejecutado cada una de las configuraciones 3 veces y hemos sacado el tiempo y colores medios para rellenar de la forma más precisa posible la tabla.

\begin{center}
\begin{tabular}{|c|c|c|}
    \hline
    Tamaño grafo & Número de colores & Runtime(s) \\
    \hline 
     N=4 ,M=6 & 2.66 & 0.1\\
     N=6 ,M=15 & 4 & 0.11\\
     N=8 ,M=28 & 5.66 & 0.115\\
     N=10 ,M=45 & 7.33 & 0.145\\
     N=12 ,M=66 & 8.66 & 0.756\\
     N=14 ,M=91 & 11.33 & 348.3\\
     \hline
\end{tabular}
\end{center}

A la vista de los resultados en tiempo de ejecución obtenidos, podemos responder a la pregunta de \textbf{¿Es este problema escalable?}.
No, este problema no va a ser practicable para grafos con muchos nodos y aristas, ya que el tiempo de ejecución volvería imposible obtener un resultado.


\end{document}